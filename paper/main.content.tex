\chapter{Uvod}\label{cahp:uvod}
S problemi velikih dimenzij sem med drugim srečujemo na različnih področjih, kot so telekomunikacije, logistika, strojništvo, itd.
Ti problemi zahtevajo iskanje optimalnih rešitev v obsežnem iskalnem prostoru, kjer ima problem veliko število odločitvenih spremenljivk.
Pri takih problemih tradicionalni optimizacijski algoritmi, kot so to metode, ki temeljijo na gradientu, kvadratno programiranje in linerano programiranje, se soočajo s težavami.
Inteligenca rojev (ang. \textit{Swarm inteligence} (SI)) se je pojavila, kot obetaven pristop za reševanje problemov velikih dimenzij zaradi svoje zmožnosti učinkovitega raziskovanja prostora, obvladovanja šumnih in ne-konveksnih objektivnih funkcij, ter obvladovanja velikega števila parametrov objektivne funkcije.

Inteligenca rojev se nanaša na kolektivno vedenje decentraliziranih, samoorganiziranjih sistemov, sestavljenih iz več agentov, ki sodelujejo med seboj in v okolju, da bi dosegli skupni clij.
Agenti v sistemih SI so lahko umetni ali naravni, komunicirajo in usklajujejo pa se po preprostih lokalnih pravilih.
Vedenje teh agentov se zgleduje po vedenju družbenih žuželk, kot so mravlje, čebele in termiti, ki v svojih kolonijah kažejo izjemne sposobnosti reševanja problemov.

Eden najbolj priljubljenih SI algoritmov za reševanje problemov velikih dimenzij je algoritem roja delcev (ang. \textit{Particle Swarm Optimization} (PSO))~\cite{alg:pso}.
PSO algoritem sta prvič predlagala Ebarhart in Kennedy leta 1995 in je od takrat pridobil široko pozornost zaradi svoje preprostosti, robustnosti in učinkovitosti.
PSO algoritem deluje tako, da simulira vedenje skupine delcev, ki se premikajo v iskalnem prostoru, da bi našli optimalno rešitev.
Vsak delec predstavlja kandidatno rešitev problema, njegovo gibanje pa je odvisno od njegove lastne hitrosti in najboljše rešitve, ki jo je skupina delcev doslej našla.

Druga SI algoritma, ki je sta lahko uporabljen za optimizacijo velikih dimenzij sta algoritem optimizacije kolonije mravelj (ang. \textit{Ant Colony Optimization} (ACO))~\cite{alg:aco} pri iskanju hrane in algoritem umetne čebelje kolonije (ang. \textit{Artificial Bee Colony} (ABC))~\cite{alg:abc}, ki posnema vedenje iskanja hrane čebel.
ACO algoritem deluje tako, da zgradi graf rešitve, ki predstavlja prostor iskanja, mravlje se premikajo po tem grafu, kjer preiskujejo prostor in iščejo nove boljše rešitve problema.
ABC algoritem zaposluje skupino čebel, ki iščejo iskalni prostor za vire hrane, pri čemer vsaka čebela predstavlja kandidatno rešitev problema.

Glavna prednost uporabe SI algoritmov za optimizacijo problemov velikih dimenzij je njihova sposobnost obvladovanja kompleksnosti in velike dimenzionalnosti iskalnega prostora.
Tradicionalni optimizacijski algoritmi se soočajo s težavami pri reševanju takih problemov zaradi visokih računskih stroškov in prisotnosti večih lokalnih optimumov.
Nasprotno pa SI algoritmi uporabljajo preporsta lokalna pravila za učinkovito raziskovanje iskalnega prostora in so odpornejši na lokalne optimume, v katerih lahko obtičijo mnogi tradicionalni algoritmi.

Uporabo SI algoritmov za optimizacijo problemov velikih dimenzij, lahko opazimo na mnogih področjih, kot so strojništvo, finance in logistika.
V strojništvu so bili SI algoritmi uporabljeni za optimizacijo kompleksnih sistemov, kot so letala, turbine in motorji.
V finačništvu so bili SI algoritmi uporabljeni za optimizacijo portfelja in obvladovanje tvegan podjetji.
V logistiki so bili SI algoritmi uporabljeni za optimizacijo usmerjanja vozil, upravljanja s skladiščem in optimizacije dobavne verige.
SI algoritmi so se izkazali za zelo učinkovite in njihov uspeh je spodbudil nadaljnje raziskave razvoja novih in izboljšanih algoritmov.

Kljub prednostim SI algoritmov imajo tudi ti algoritmi nekatere omejitve, kot so počasna stopnja konvergence, potreba po velikem številu ovrednotenj cenitvene funkcije in občutljivosti na nastavitve parametrov algoritma.
Za premagovanje teh omejitev so raziskovali predlagali več modifikacij in hibridizacij SI algoritmov, kot je hibridizacija z lokalnimi iskalnimi metodami, operatorji mutacije in hibridizacija z drugimi optimizacijskimi algoritmi.

V zadnjih letih so raziskovalci uvedli koncept diferencialnega združevanja in koevolucije kot nove strategije za izboljšanje učinkovitosti optimizacijskih algoritmov za reševanje problemov velikih dimenzij.
Cilj teh strategij je izboljšati proces iskanja z razdelitvijo optimizacijskega problema na manjše podprobleme in sočasno uporabo optimizacijskih algoritmov za optimizacijo vsakega podproblema.
Podproblemi se nato uskladijo z diferencialnim združevanjem in koevoluciskimi mehanizmi, na način da vodijo celoten proces iskanja k boljšim rešitvam.

Koevolucija je tehnika, ki podpira reševanje hkraten razvoj večih populacij, ki rešujejo svoj podproblem.
Podproblemi medsebojno delujejo in vplivajo na razvoj drug drugega, ter ustvarjajo dinamično in konkurenčno okolje.
Koevolucija spodbuja raziskovanje različnih regij iskalnega prostora in spodbuja izmenjavo informacij med podproblemi.
Ta kooperativna interakcija med podproblemi lahko vodi do odkritja boljših rešitev in izboljšanja konvergentnih lastnosti.

Kombinacija diferencialnega združevanja in koevelucije za hribidizacijo optimizacijskih algoritmov je pokazala obetavne rezultate pri reševanju problemov velikih dimenzij.
Te strategije izkoriščajo prednosti obeh tehnik, kar omogoča učinkovito raziskovanje in izkoriščanje iskalnega prostora.
Z razčlenitvijo problema na manjše podprobleme in uporabo optimizacijskih algoritmov postane proces iskanja bolj osredotočen in učinkovit.

Motivacija za to diplomsko delo izhaja iz potrebe po obravnavanju naraščajočega povpraševanja po učinkovitih in uspešnih rešitvah problemov velikih dimenzij.
S pojavom vele podatkov (ang. \textit{Big Data}) in vse večjo kompleksnostjo sodobnih sistemov, tradicionalne optimizacijske tehnike ne zadoščajo za pravočasno zagoravljanje optimalnih rešitve.
PSO algoritmi ponujajo obetavno alternativo za reševanje teh zahtevnih problemov, vendar so potrebne nadaljnje raziskav, da bi v celoti razumeli njihove zmogljivosti in omejitve.

V tem diplomskem delu želimo raziskati učinkovitost kovelucijskega hibridnega PSO algoritma z rekurzivnimi diferencialnimi metodami grupiranja za reševanje problemov velikih dimenzij.
Raziskali bomo, kako lahko te strategije izboljšajo delovanje PSO algoritmov.
Z obsežnim eksperimentiranjem in primerjalno analizo bomo ovrednotili delovanje predlaganega algoritma v primerjavi z obstoječimi hibridnimi PSO algoritmi.
S preučevanjem njegove hitrosti konvergence, kakovosti rešitev in robustnosti želimo zagotoviti vpogled v potencialne prednosti in omejitve novega hibridnega algoritma v kontekstu problemov velikih dimenzij.
Z vključitvijo konceptov diferencialnega grupiranja in koevolucije v hibridne PSO algoritme pričakujemo, da bomo premagali nekatere omejitve in izzive, ki jih predstavljajo optimizacijski problemi velikih dimenzij.
Te strategije imajo potencial za izboljšanje zmožnosti raziskovanja in izkoriščanja optimizacijskih algoritmov,  kar vodi do učinkovitejših in uspešnejših rešitev.
Preiskava teh tehnik v kontekstu problemov veliki dimenzij je ključnega pomena za napredek na področju optimizacijskih algoritmov iz zagotavljanja praktičnih rešitev za aplikacije v resničnem svetu.

Prispevki tega diplomskega dela naj bi vključevali:
\begin{enumerate}
    \item Obsežen pregled literature o PSO algoritmih, vključno zz analizo prednosti in omejitev različnih PSO algoritmov.
    \item Vrednotenje delovanja različnih PSO algoritmov na problemih velikih dimenzij, ki vključujejo multimudalne funkcije, popolnoma odvedljive funkcije, delno aditivno odvedljive funkcije, funkcije s prekrivajočimi komponentami in ne-odvedljive funkcije.
    \item Raziskovanje vpliva različnih uporabljenih metod hibridizacije PSO algoritma na problemih velikih dimenzij.
    \item Priporočila za izbiro in nastavitve hibridiziranih PSO algoritmov za probleme velikih dimenzij, ki temeljijo na rezultatih naših poskusov.
\end{enumerate}

Preostanek tega diplomskega dela je organiziran na naslednji način.
V \ref{chap:obsojeca.dela}. poglavju ponujamo pregled literature o optimizacijskih algoritmih inteligence rojev za optimizacijo problemov velikih dimenzij.
To poglavje vključuje podrobno razpravo o prednostih in omejitvah različnih optimizacijskih algoritmov inteligence rojev, algoritmov grupiranja komponent, ki bazirajo na razlikah vpliva komponent, ter koevolucijskih algoritmih za reševanje problemov velikih dimenzij.
V \ref{chap:algo}. poglavju opisujemo novo hibidizacijo PSO algoritma, ki uporablja koevolucijo in grupiranje komponent za probleme velikih dimenzij.
To poglavje vključuje razpravo o uporabljenih metodah za modifikacijo hibridnega PSO algoritma, ter omejitvah novega hibridnega algoritma.
Poglavje vsebuje tudi priporočila za izbiro, nastavitev parametrov in omejitve hibridnih PSO algoritmov za probleme velikih dimenzij.
V \ref{chap:exp}. poglavju posredujemo rezultate naših poskusov in razpravljamo o uspešnosti različnih hibridnih PSO algoritmih na problemih velikih dimenzij.
To poglavje vključuje podrobno analizo vpliva različnih parametrov hibridnih PSO algoritmov.
Na koncu \ref{chap:end}. poglavju podajamo povzetek naših ugotovitev, zaključkom in predlogi za prihodnje raziskave na tem področju. Verjamemo, da bo ta diplomska naloga zagotovila dragocen vpogled v uporabo SI algoritmov za optimizacijo problemov velikih dimenzij in prispevala k razvoju učinkovitejših in uspešnejših optimizacijskih tehnik.

\chapter{Pregled obstoječih del}\label{chap:obsojeca.dela}

% TODO

\section{Optimizacijski algoritmi in inteligenca rojev}

% TODO

\section{Roj delcev}

% TODO

\section{Ko-evolucija}

% TODO

\section{Diferencialne metode grupiranja}

% TODO

\begin{algorithm}
    \KwIn{$f$, $\mathbf{ub}$, $\mathbf{lb}$, $\alpha$, $k$}
    \SetKwFunction{Inter}{Interact}
    \SetKwFunction{app}{Append}
    \SetKwFunction{del}{Delete}

    $\mathbf{a}$, $\mathbf{b}$ \gets~ [], []\;
    $\epsilon$ \gets~ $\alpha \cdot \min(|f(\mathbf{x}_1|, \cdots, f(\mathbf{x}_k|)$\;
    $\mathbf{x}_{l,l}$, $y_{l,l}$ \gets~ $\mathbf{lb}$, $f(\mathbf{lb})$\;
    $\mathbf{x}_r$, $\mathbf{s}_1$, $\mathbf{s}_2$ \gets~ [1], [1], [\lFor*{$i=2$ \KwTo \ell($\mathbf{lb}$)}{$i$}]\;

    \While{\ell($\mathbf{x}_r$) > 0}{
        $\mathbf{x}_r$ \gets~ []\;
        $\mathbf{s}^*$ \gets~ \Inter{$f$, $\mathbf{x}_{l,l}$, $y_{l,l}$, $\epsilon$, $\mathbf{s}_1$, $\mathbf{s}_2$, $\mathbf{x}_r$}\;
        \eIf{\ell($\mathbf{s}^*$) $==$ \ell($\mathbf{s}_1$)}{
            \leIf{\ell($\mathbf{s}_1$) $=$ $1$}{$\mathbf{a}$ \gets~ $\mathbf{a} \cup \mathbf{s}_1$}{\app{$\mathbf{b}$, $\mathbf{s}_1$}}
            \eIf{\ell(\app{$\mathbf{x}_r$} > $1$)}{
                $\mathbf{s}_1$ \gets~ $\mathbf{x}_r$[:1]\;
                $\mathbf{x}_r$ \gets~ $\mathbf{x}_r$[1:]\;
                $\mathbf{s}_2$ = $\mathbf{x}_r$\;
            }{
                $\mathbf{a}$ \gets~ $\mathbf{a} \cup \mathbf{s}_1$\;
                break\;
            }
        }{
            $\mathbf{s}_1$ \gets~ $\mathbf{s}^*$\;
            $\mathbf{s}_2$ \gets~ $\mathbf{x}_r$\;
            \If{\ell($\mathbf{s}_r$) = $0$}{
                \app{$\mathbf{b}$, $\mathbf{s}_1$}\;
                break\;
            }
        }
    }

    \lForEach{$e$ in $\mathbf{a}$}{\app{$\mathbf{b}$, $[e]$}}
    \Return $\mathbf{b}$

    \caption{\textbf{Algoritem RDG za dekompozicijo problema.}}
    \label{pcode:rdg3}
\end{algorithm}


\begin{algorithm}
    
	\KwIn{$f$, $\mathbf{lb}$, $\mathbf{ub}$, $\mathbf{x}_{l,l}$, $y_{l,l}$, $\epsilon$, $\mathbf{s}_1$, $\mathbf{s}_2$, $\mathbf{x}_r$}
	\SetKwFunction{Inter}{Interact}
    \SetKwFunction{floor}{floor}

    $\mathbf{x}_{u,l}$, $\mathbf{x}_{l,m}$ \gets~ $\mathbf{x}_{l,l}$, $\mathbf{x}_{l,l}$\;
    $\mathbf{x}_{u,m}$, $\mathbf{x}_n$ \gets~ $\mathbf{x}_{u,l}$, $\mathbf{s}_1$\;
    $\mathbf{x}_{u,l}$, $\mathbf{x}_{l,m}$, $\mathbf{x}_{u,m}$ \gets~ $\mathbf{lb}$[$\mathbf{s}_1$], $(\mathbf{lb} + \mathbf{ub}) / 2)$[$\mathbf{s}_2$], $(\mathbf{lb} + \mathbf{ub}) / 2)$[$\mathbf{s}_2$]\;
    $y_{u,l}$, $y_{l,m}$, $y_{u,m}$ \gets~ $f(\mathbf{x}_{u,l})$, $f(\mathbf{x}_{l,m})$, $f(\mathbf{x}_{u,m})$\;
    $\sigma_1$, $\sigma_2$ \gets~ $y_{l,l} - y_{u,l}$, $y_{l,m} - y_{u,m}$\;

    \If{$|\sigma_1 - \sigma_2| > \epsilon$}{
        \eIf{\ell($\mathbf{s}_2$) $=$ $1$}{
            $\mathbf{x}_n$ \gets~ $\mathbf{s}_1 \cup \mathbf{s}_2$\;
        }{
            $k$ \gets \floor{\ell($\mathbf{s}_2$) / 2}\;
            $\mathbf{g}_1$, $\mathbf{g}_2$ \gets~ $\mathbf{s}_2$[:$k$], $\mathbf{s}_2$[$k$:]\;
            $\mathbf{x}_1$ \gets~ \Inter{$f$, $\mathbf{x}_{l,l}$, $y_{l,l}$, $\epsilon$, $\mathbf{s}_1$, $\mathbf{g}_1$, $\mathbf{x}_r$}\;
            $\mathbf{x}_2$ \gets~ \Inter{$f$, $\mathbf{x}_{l,l}$, $y_{l,l}$, $\epsilon$, $\mathbf{s}_1$, $\mathbf{g}_2$, $\mathbf{x}_r$}\;
            $\mathbf{x}_n$ \gets~ $\mathbf{x}_1 \cup \mathbf{x}_2$\;
        }
    }

    \Return $\mathbf{x}_n$
 
	\caption{\textbf{Funkcija za detekcijo interakcije med komponentami.}}
	\label{pcode:interact}
\end{algorithm}

\section{Omejitve sorodnih del}

% TODO

\chapter{Koevolucijski algoritem rojev delcev}\label{chap:algo}

% TODO

\begin{algorithm}
	\KwIn{$f$, $\mathbf{lb}$, $\mathbf{ub}$}
	\SetKwFunction{run}{RunGeneration}
	\SetKwFunction{Decom}{Decomposition}
	\SetKwFunction{appbetter}{Update}

	$\mathbf{g}$ \gets~ \Decom{$f$, $\mathbf{ub}$, $\mathbf{lb}$}\;
	$\mathbf{a}$ \gets~ [\lForEach*{$g$ in $\mathbf{g}$}{Initialize algorithm $a$ with starting population based on group $g$}]\;
        $\mathbf{X^*}$, $\mathbf{y}^*$ \gets~ Get best individuals and fitness values from initialized populations for each algorithm\;\label{pcode:ccalgo:get_bests}
	$\mathbf{x^*}$, $y^*$ \gets~ Get best individual and fitness value from $\mathbf{X^*}$\;
	\While{$\neg$ stopping condition meet}{
		\ForEach{$i$, $a$ in $\mathbf{a}$}{
			$\mathbf{x_n^*}$, $y_n^*$ \gets~ \run{$a$}\;\label{pcode:ccalgo:run_gen}
			\If{$y_n^* < \mathbf{y^*}[i]$}{\label{pcode:ccalgo:best_check}
				$\mathbf{X^*}[i]$, $\mathbf{y^*}[i]$ \gets~ $\mathbf{x_n^*}$, $y_n^*$\;\label{pcode:ccalgo:construct_best}
				\lIf{$y_n^* < y^*$}{$\mathbf{x^*}$, $y^*$ \gets~ $\mathbf{x_n^*}$, $y_n^*$}
			}
		}
        \If{Any group found new best individual}{\label{pcode:ccalgo:new_best_fond_start}
		      \lForEach{$g$ in $\mathbf{g}$}{$\mathbf{x_n}[g]$ \gets~ $\mathbf{X^*}[g]$}
		      $y_n$ \gets~ $f(\mathbf{x_n})$\;
		      \lIf{$y_n < y^*$}{$\mathbf{x^*}$, $y^*$ = $\mathbf{x_n}$, $y_n$}
        }\label{pcode:ccalgo:new_best_fond_end}
	}
	\Return{$\mathbf{x^*}$, $y^*$}\;

	\caption{\textbf{Our implementation of CC algorithm.}}
	\label{pcode:ccalgo}
\end{algorithm}

\chapter{Eksperiment}\label{chap:exp}

% TODO

\section{Primerjava uporabe metod RDG}

% TODO

\subsection{Analiza rezultatov}

% TODO

\section{Primerjava PSO algoritmov z metodami RDG}

% TODO

\subsection{Analiza rezultatov}

% TODO

\chapter{Zaključek}\label{chap:end}

% TODO
