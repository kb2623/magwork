$\otimes$ -- Množenje vektorjev po istoležnih komponentah \\
\ell$(\cdot)$ -- Funkcija vrne število elementov vektorja oziroma seznama \\
$f(\cdot)$ -- Ocenitvena funkcija, ki predstavlja problem za reševanje \\
$\mathcal{U}(p_1, p_2)$ -- Funkcija enakomerne naključne porazdelitve, ki vrača naključne realne vrednosti v intervalu od $p_1$ do $p_2$ \\
$\mathit{D}$ -- Dimenzija problema \\
$\mathit{M}$ -- Parameter algoritma, ki predstavlja število mutacij delca \\
$\mathit{NP}$ -- Konstanta, ki predstavlja velikost populacije \\
$t$ -- Številka trenutne generacije \\
$t_{max}$ -- Maksimalno število generacij algoritma \\
$w$ -- Vztrajnostna utež pri algoritmih, ki bazirajo na algoritmu PSO \\
$w_l$ -- Realna vrednost minimalne vrednosti vztrajnostne uteži \\
$w_u$ -- Realna vrednost maksimalne vrednosti vztrajnostne uteži \\
$p_0$ -- Realna vrednost, ki predstavlja verjetnost, kdaj se algoritem OVCPSO odloči za fazo učenja z nasprotnimi delci \\
$\mathbf{b}_{l}$ -- Vektor z vrednostmi, ki predstavljajo spodnjo mejo iskalnega prostora \\
$\mathbf{b}_{u}$ -- Vektor z vrednostmi, ki predstavljajo zgornjo mejo iskalnega prostora \\
$\mathbf{o}_l$ -- Vektor z vrednostmi, ki definirajo spodnjo mejo trenutnega delca \\
$\mathbf{o}_u$ -- Vektor z vrednostmi, ki definirajo zgornjo mejo trenutnega delca \\
$\mathbf{x}_i$ -- Vektor dimenzije $\mathit{D}$, ki predstavlja položaj trenutnega delca v iskalnem prostoru \\
$\mathbf{x}_i'$ -- Nasprotni delec delca $\mathbf{x}_i$ \\
$\mathbf{X}$ -- Matrika, ki predstavlja populacijo optimizacijskega algoritma, ki je dimenzije $\mathit{NP} \times \mathit{D}$ \\
$\mathbf{x}^*$ -- Vektor, ki hrani optimalne vrednosti parametrov funkcije $f$ \\
$y^*$ -- Realna rednost, ki predstavlja ocenitveno vrednost vektorja $\mathbf{x}^*$ na podlagi funkcije $f$ \\
$\mathbf{p}_i$ -- Vektor, ki hrani najboljšo najdeno pozicijo posameznika $i$ \\
$\mathbf{P}$ -- Matrika, ki predstavlja najboljše znane pozicije delcev roja, ter je dimenzije $\mathit{NP} \times \mathit{D}$ \\
$r_1$ -- Naključno število v intervalu od $0$ do $1$ \\
$r_2$ -- Naključno število v intervalu od $0$ do $1$ \\
$\mathbf{r}_1$ -- Vektor naključnih števil dimenzionalnosti $\mathit{D}$, ki vsebuje naključna števila iz intervala od $0$ do $1$, ki so pridobljena na podlagi enakomerne naključne porazdelitve \\
$\mathbf{r}_2$ -- Vektor naključnih števil dimenzionalnosti $\mathit{D}$, ki vsebuje naključna števila iz intervala od $0$ do $1$, ki so pridobljena na podlagi enakomerne naključne porazdelitve \\
$\mathbf{r}_3$ -- Vektor naključnih števil dimenzionalnosti $\mathit{D}$, ki vsebuje naključna števila iz intervala od $0$ do $1$, ki so pridobljena na podlagi enakomerne naključne porazdelitve \\
$c_1$ -- Parameter, ki upravlja moč kognitivne komponente pri izračunu hitrosti delca \\
$c_2$ -- Parameter, ki upravlja moč socialne komponente pri izračunu hitrosti delca \\
$\alpha$ -- Parameter algoritmov, ki izvirajo iz algoritma RDG, za nadzor parametra $\epsilon$  \\
$\epsilon$ -- Mejna vrednost za detekcijo interakcije skupine komponente skupini s skupino komponent \\
$\Omega$ -- Dopustni iskalni prostor \\
$k$ -- Parameter algoritmov RDG in TRDG, za kontrolo vrednosti $\epsilon$ \\
$\epsilon_e$ -- Parameter algoritma RDG3 za nadzor velikosti skupin komponent \\
$\epsilon_n$ -- Parameter algoritma RDG3 za nadzor velikosti skupin komponent \\
\\
SI -- Inteligenca rojev (angl. \textit{Swarm Inteligence}) \\
CI -- Računska inteligenca (angl. \textit{Computational Intelligence}) \\
PSO -- Algoritem roja delcev (angl. \textit{Particle Swarm Optimization}) \\
ACO -- Algoritem optimizacije kolonije mravelj (angl. \textit{Ant Colony Optimization}) \\
ABC -- Algoritem umetne čebelje kolonije (angl. \textit{Artificial Bee Colony}) \\
LDWPSO -- Algoritem roja delcev z utežjo za linearno zmanjševanje hitrosti delca (angl. \textit{Linearly Decreasing Weight Particle Swarm Optimization}) \\
CPSO -- Algoritem roja delcev s centralnim delcem (angl. \textit{Center Particle Swarm Optimization}) \\
HPSO -- Hibridni algoritem roja delcev (angl. \textit{Hybrid Particle Swarm Optimization}) \\
MCPSO -- Algoritem roja delcev z dodano mutacijo delcev in centralnim iskanjem (angl. \textit{Mutated Center Particle Swarm Optimization}) \\
CLPSO -- Algoritem roja delcev s celovitim učenjem (angl. \textit{Comprehensive Learning Particle Swarm Optimizer}) \\
UPS -- Algoritem enotnega roja delcev (angl. \textit{Unified Particle Swarm}) \\
OVCPSO -- Algoritem roja delcev z uporabo nasprotnih delcev in omejevanjem hitrosti (angl. \textit{Opposition-Based Particle Swarm Optimization with Velocity Clamping}) \\
LSGO -- Globalna optimizacija problemov z velikim številom dimenzij (angl. \textit{Large Scale Global Optimization}) \\
RDG -- Rekurzivno diferencialno grupiranje (angl. \textit{RecursiveDifferentialGrouping}) \\
RDGV2 -- Rekurzivno diferencialno grupiranje verzije 2 (angl. \textit{Recursive Differential Grouping V2}) \\
RDGV3 -- Rekurzivno diferencialno grupiranje verzije 3 (angl. \textit{Recursive Differential Grouping V3}) \\
ERDG -- Efektivno rekurzivno diferencialno grupiranje (angl. \textit{Efficient Recursive Differential Grouping}) \\
TRDG -- tri nivojsko rekurzivno diferencialno grupiranje (angl. \textit{Three Level Recursive Differential Grouping}) \\
BD -- Vele podatki (angl. \textit{Big Data}) \\
BB -- Črna škatla (angl. \textit{Black Box}) \\
CC -- Kooperativna koevolucija (angl. \textit{Cooperative Coevolution}) \\
CCGA -- Kooperativni koevolucijski genetski algoritem (angl. \textit{Cooperative Co-evolutionary Genetic Algorithm}) \\
GA -- Genetski algoritem (angl. \textit{Genetic Algorithm}) \\
NPSO-CC -- Nov roj delcev s kooperativno koevolucijo (angl. (New Particle Swarm Optimizer with Cooperative Coevolution))\\